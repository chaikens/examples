\documentclass{article}
\usepackage{hyperref}
\setlength{\textwidth}{6.5in}
\setlength{\oddsidemargin}{0.0in}
\setlength{\evensidemargin}{0.0in}
\setlength{\textheight}{9in}
\setlength{\topmargin}{0.0in}
\setlength{\headheight}{0.0in}
\setlength{\headsep}{0.0in}

\begin{document}
\begin{center}
ICSI403 Algorithms Spring 2011 Chaiken\hfill
Homework 1\hfill
Assigned 1/20\hfill
Due 1/27\\
Note: There will be 3 kinds of HW problems: Theory (T), Programming (P) and
Experiments (E).
\end{center}

\begin{itemize}

\item[T1.] \textbf{(Submit on PAPER)}
CLRS Exercise 2.1-1 (Illustrate INSERTION-SORT).
\item[P1.] \textbf{(Submit on BLACKBOARD)}
Add to my \texttt{SortableArray} Java program your implementation
of INSERTION-SORT as described in CLRS.  Download the .java files for the
\texttt{SortableArray} and \texttt{Stopwatch} classes from the web site
\url{http://www.cs.albany.edu/~sdc/CSI403}.  But use
subscripts starting with zero rather than 1 as in CLRS.
\end{itemize}

\textbf{All} assignments (this and future)
to modify a program and/or create your own
program must meet the following requirements for credit:
\begin{enumerate}
\item All the .java files (including copies you have not modified)
needed to compile and demonstrate your work must be submitted on Blackboard.
\item The .java files together must be compilable (with no errors!) when
the command ``\verb|javac *.java|'' is given in a directory containing them
all, after any submitted .class files have been removed.  
The problem gets zero credit if compiling fails! (In a senior level CS 
course, you are expected to be able to find and correct syntax and
missing class errors when they are detected by the compiler and
the JVM!)
\item Your submission must include one or more text files, documents 
(preferably in .pdf), screen shots, etc. that report how you tested all
the functionality of your code.  For full credit, tests must be reported
of trivial problem sizes, small sizes and sizes large enough to be generic.
Similarly, if input data is used, tests on both data for special cases and
generic data must be done.  (Examples of special case data for sorting are
a file of already sorted numbers and a file of many copies of the same number.)
\item One of the files submitted must be named ``ReadMeFirst'' (with the
appropriate extension) and it must guide who's grading your work how to
evaluate it.  If you like, you can put all your test reports in the 
``ReadMeFirst'' document.  Hint: Keep a written journal document
while you work, from
beginning to end.  Then, when you are done assuring yourself that your 
code is bug-free, your report will be finished!
\item Faked test reports are considered to be \textbf{cheating}!  
If you cannot complete a correct program, submitting a compilable
and testable program with bugs plus a report of your progress, including
where you got stuck, will earn partial credit.
\end{enumerate}


\begin{itemize}
\item[T2.] \textbf{(Submit on PAPER)}
CLRS Exercise 2.3-1 (Illustrate MERGE-SORT).
\item[E1.] \textbf{(Submit on PAPER)}
For my two sorting algorithm implementations, plus your
implementation of insertion-sort if you finish it (correctly!), gather
timing results for input sizes in the 1, 2, 5, 10, 20, 50, 100, 200, ... 
series up to the largest sizes your program can handle in a reasonable amount
of time.  Specifically, do timings for sizes up to those that either crash 
your Java run or take more than 10 minutes.  Plot all results on linear
plus log-log graph paper so that they are clearly presented both for 
small and for large problem sizes.  A linear plot for times up to 1 second
should be included.  

Do all the timings on the same computer, with the computer running as
few other programs as possible.  Include a report of the model of
computer processor, sizes of caches, clock frequency (in Gigahertz),
RAM size, OS, etc., plus the version of the Java compiler and the Java
virtual machine (JVM).  (You can get that information by running the
commands ``\verb|javac -version| and ``\verb|java -showversion|'').
Also report in detail if you increased the memory used by the JVM.

Graph paper is supplied and you can print more from the course web site.

\item[T3.] \textbf{(Submit on PAPER)}
CLRS Exercise 2.3-7 ($\Theta(n \log n)$ algorithm to tell if
some two numbers in a length $n$ array add up to $x$) Hint: First sort the
array. You will get 80\% of the problem credit if you turn in a few
pages of scratch work in which you try out algorithmic ideas on examples of
sorted arrays.  The remaining 20\% will be awarded for a clear and correct
algorithm description in English and/or pseudo-code together with an
explanation of why it works.
\end{itemize}
\end{document}
